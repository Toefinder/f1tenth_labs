\documentclass[letta4 paper]{article}
% Set target color model to RGB
\usepackage[inner=2.0cm,outer=2.0cm,top=2.5cm,bottom=2.5cm]{geometry}
\usepackage{setspace}
\usepackage[rgb]{xcolor}
\usepackage{verbatim}
\usepackage{subcaption}
\usepackage{amsgen,amsmath,amstext,amsbsy,amsopn,tikz,amssymb,tkz-linknodes}
\usepackage{fancyhdr}
\usepackage[colorlinks=true, urlcolor=blue,  linkcolor=blue, citecolor=blue]{hyperref}
\usepackage[colorinlistoftodos]{todonotes}
\usepackage{rotating}
\usepackage{listings}
\usepackage{algorithm}
\usepackage{algorithmic}
\lstset{
%	language=bash,
	basicstyle=\ttfamily
}

\newcommand{\ra}[1]{\renewcommand{\arraystretch}{#1}}

\newtheorem{thm}{Theorem}[section]
\newtheorem{prop}[thm]{Proposition}
\newtheorem{lem}[thm]{Lemma}
\newtheorem{cor}[thm]{Corollary}
\newtheorem{defn}[thm]{Definition}
\newtheorem{rem}[thm]{Remark}
\numberwithin{equation}{section}
\graphicspath{ {./img/} }

\newcommand{\homework}[6]{
   \pagestyle{myheadings}
   \thispagestyle{plain}
   \newpage
   \setcounter{page}{1}
   \noindent
   \begin{center}
   \framebox{
      \vbox{\vspace{2mm}
    \hbox to 6.28in { {\bf F1TENTH Autonomous Racing \hfill {\small (#2)}} }
       \vspace{6mm}
       \hbox to 6.28in { {\Large \hfill #1  \hfill} }
       \vspace{6mm}
       \hbox to 6.28in { {\it Instructor: {\rm #3} \hfill Name: {\rm #5}, StudentID: {\rm #6}} }
       %\hbox to 6.28in { {\it T\textbf{A:} #4  \hfill #6}}
      \vspace{2mm}}
   }
   \end{center}
   \markboth{#5 -- #1}{#5 -- #1}
   \vspace*{4mm}
}


\newcommand{\problem}[3]{~\\\fbox{\textbf{Problem #1: #2}}\hfill (#3 points)\newline}
\newcommand{\subproblem}[1]{~\newline\textbf{(#1)}}
\newcommand{\D}{\mathcal{D}}
\newcommand{\Hy}{\mathcal{H}}
\newcommand{\VS}{\textrm{VS}}
\newcommand{\solution}{~\newline\textbf{\textit{(Solution)}} }

\newcommand{\bbF}{\mathbb{F}}
\newcommand{\bbX}{\mathbb{X}}
\newcommand{\bI}{\mathbf{I}}
\newcommand{\bX}{\mathbf{X}}
\newcommand{\bY}{\mathbf{Y}}
\newcommand{\bepsilon}{\boldsymbol{\epsilon}}
\newcommand{\balpha}{\boldsymbol{\alpha}}
\newcommand{\bbeta}{\boldsymbol{\beta}}
\newcommand{\0}{\mathbf{0}}


\usepackage{booktabs}



\begin{document}

	\homework {Lab 1: Introduction to ROS}{Due Date: Aug 9th 2021}{Yuhas Michael John}{}{Le Quang Anh}{U1922940J}
	\thispagestyle{empty}
	% -------- DO NOT REMOVE THIS LICENSE PARAGRAPH	----------------%
	\begin{table}[h]
		\begin{tabular}{l p{14cm}}
		\raisebox{-2cm}{\includegraphics[scale=0.5, height=2.5cm]{f1_stickers_02} } & \textit{This lab and all related course material on \href{http://f1tenth.org/}{F1TENTH Autonomous Racing} has been developed by the Safe Autonomous Systems Lab at the University of Pennsylvania (Dr. Rahul Mangharam). It is licensed under a \href{https://creativecommons.org/licenses/by-nc-sa/4.0/}{Creative Commons Attribution-NonCommercial-ShareAlike 4.0 International License.} You may download, use, and modify the material, but must give attribution appropriately. Best practices can be found \href{https://wiki.creativecommons.org/wiki/best_practices_for_attribution}{here}.}
		\end{tabular}
	\end{table}
	% -------- DO NOT REMOVE THIS LICENSE PARAGRAPH	----------------%
	
	\noindent \large{\textbf{Course Policy:}} Read all the instructions below carefully before you start working on the assignment, and before you make a submission. All sources of material must be cited. The University Academic Code of Conduct will be strictly enforced.

	\section{Workspaces and Packages}

	
	\subsection{Written Questions}
	\begin{enumerate}
		\item CMakeList file contains a set of directives and instructions describing the project's source files and targets. 
		It is the input to the CMake build system for building software packages. Any CMake-compliant package contains one or more CMakeLists.txt file that describe how to build the code and where to install it to.
		
		Makefile is a simple way to organise code compilation for C/C++ objects.
		
		CMakeList is related to makefile. Make (or rather a Makefile) is a buildsystem. 
		It drives the compiler and other build tools to build your code. CMake is a generator of buildsystems. It can produce Makefiles, it can produce Ninja build files, ect (Reference:
		\href{https://stackoverflow.com/questions/25789644/difference-between-using-makefile-and-cmake-to-compile-the-code}{stackoverflow}).
	 
		\item Yes, we also use CMakeLists.txt for Python in ROS

		There are no executable object created for python. Python is an interpreted language.
		\item We run \lstinline {catkin_make} in the workspace folder to build all the packages
		\item Sourcing \textit{setup.bash} file is so that we can setup the environment, because ROS relies on the notion of combining spaces using the shell environment 
		(Reference: \href{http://wiki.ros.org/ROS/Tutorials/InstallingandConfiguringROSEnvironment}{ROS Wiki}).
		
	\end{enumerate}{}

	\section{Publishers and Subscribers}
	
	\subsection{Written Questions}
	\begin{enumerate}
		\item Answer here
		\item There is no nodehandle object in Python. \lstinline{rospy.init_node()} initializes ROS node with a specified name for the rospy process.
		\item We use \lstinline{rospy.spin()} for Python. The command stays in an infinite loop until receiving a shutdown signal, and processes any events that occur (Reference: \href{https://answers.ros.org/question/332192/difference-between-rospyspin-and-rospysleep/}{ROS Forum}). The node with the subscriber will do the callback whenever it receives new data from the topic, until it receives a shutdown signal.
		\item Answer here
		\item Answer here
	\end{enumerate}{}
			
	\section{ Implementing Custom Messages}
	
	\subsection{Written Questions}
	\begin{enumerate}
		\item Answer here
		\item `Header` is a special type in ROS, which contains a timestamp and coordinate frame information that are commonly used in ROS (Reference: \href{http://wiki.ros.org/ROS/Tutorials/CreatingMsgAndSrv#Creating_a_msg}{ROS Wiki}). I can also extract that from the topic and include in the message file. 
	\end{enumerate}{}

	\section{Recording and Publishing Bag Files}
	\subsection{Written Questions}
	
	\begin{enumerate}
		\item The bag file gets saved in the current directory. To change where it is saved, we need to add the path to the \lstinline{-o} argument when calling \lstinline{rosbag record}.
		\item By default, it will be in the \textit{~/.ros} folder. We can change where it is saved by editing the \lstinline{args} for the node in the launch file (Reference: \href{https://answers.ros.org/question/106126/giving-a-directory-to-output-file-in-a-launcher-for-rosbag-record-and-rostopic-echo/?answer=222874?answer=222874#post-id-222874}{ROS Forum}).
	\end{enumerate}{}




\end{document} 
